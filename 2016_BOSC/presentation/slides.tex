%\documentclass[trans,handout]{beamer}
\documentclass[trans]{beamer}
\usetheme{eit}

%\usepackage{pgfpages}
%\pgfpagesuselayout{4 on 1}[a4paper,landscape,border shrink=5mm]
%\pgfpagesuselayout{2 on 1}[a4paper,border shrink=5mm]

\usepackage[latin1]{inputenc}
\usepackage{graphicx}
\usepackage{textpos}
\usepackage{textcomp}
\usepackage{floatrow}
\usepackage{hyperref}
\hypersetup{
  colorlinks=true,
  linkcolor=black,
  urlcolor=blue
}
\graphicspath{{figures/}}

\title{
  \includegraphics[height=.2\textheight]{../abstract/biopython.jpg}\\[1em]
  Biopython Project Update 2016}
\subtitle{}
\author{
  \textbf{Christian Brueffer}*, Tiago Ant\~{a}o, Peter Cock, Eric Talevich,\\
  Michiel de Hoon, Wibowo Arindrarto, Leighton Pritchard,\\
  Anuj Sharma, Eric Rasche, Aaron Rosenfeld, Connor T.\\
  Skennerton, Marco Galardini, Markus Piotrowski,\\
  and the Biopython Contributors}
\institute[Translational Oncogenomics Unit, Department of Clinical Sciences, Lund University]{* Translational Oncogenomics Unit\\Department of Clinical Sciences \\
  Lund University\\
  Sweden\\[1em]
  Bioinformatics Open Source Conference 2016, Orlando, USA \\[1em]
  %\includegraphics[height=.2\textheight]{../abstract/biopython.jpg}
}
\date{July 9th, 2016}


\setcounter{tocdepth}{2}
\setbeamertemplate{caption}{\insertcaption}


% ToC at the beginning of every section
%\AtBeginSection[]
%{
 % \begin{frame} % with <beamer> => doesn't appear in handout mode
  %  \frametitle{Outline} %% Put the title you want, or none!
  %  \tableofcontents[currentsection,currentsection]
 % \end{frame}
%}

\begin{document}

%%%%%%%%%%%%%%%%%%%%%%%%%%%%%%%%%%%%%%%%%%%%%%%%%%%%%%%%%%%%%%%%%%%%%%%%%%%%%%%%
\begin{frame}
	\titlepage
\end{frame}
\setbeamertemplate{footline}[body]

\begin{frame}
    \frametitle{Outline}
    \tableofcontents
\end{frame}

%%%%%%%%%%%%%%%%%%%%%%%%%%%%%%%%%%%%%%%%%%%%%%%%%%%%%%%%%%%%%%%%%%%%%%%%%%%%%%%%

\section{The Biopython Project}
\frame
{
  \frametitle{What is Biopython?}

  \begin{itemize}
  \item Collection of modules for biological computation in Python
  \item First release in 2000
  \item Open source and freely available (Biopython license)
  \end{itemize}
}
\frame
{
  \frametitle{Blackduck OpenHUB Stats}

}

%%%%%%%%%%%%%%%%%%%%%%%%%%%%%%%%%%%%%%%%%%%%%%%%%%%%%%%%%%%%%%%%%%%%%%%%%%%%%%%%

\section{Releases}
\subsection*{Release 1.66}
\frame
{
  \frametitle{New stuff in 1.66}

  \begin{itemize}
  \item extended Bio.KEGG and Bio.Graphics modules to support drawing KEGG pathways with transparency
  \item extended "abi" Bio.SeqIO parser to decode almost all documented fields used by ABIF instruments
  \item QCPSuper-imposer module using the Quaternion Characteristic Polynomial algorithm for superimposing structures to Bio.PDB
  \item extended Bio.Entrez module to implement the NCBI Entrez Citation Matching function and to support NCBI XML files with XSD schemas.
  \item miscellaneous bug fixes, enhanced test suite, better PEP8 coding style adherence
  \end{itemize}
}
\subsection*{Release 1.67}
\frame
{
  \frametitle{New Stuff in 1.67}

  \begin{itemize}
  \item deprecate the ability to compare SeqRecord objects with "==" -- sometimes lead to surprising results
  \item new experimental Bio.phenotype module for working with Phenotype Microarray data updates to Bio.Data to include NCBI genetic code table 25, covering Candidate Division SR1 and Gracilibacteria
  \item update to Bio.Restriction to include the REBASE May 2016 restriction enzyme list
  \item updates to BioSQL to use foreign keys with SQLite3 databases
  \item corrections to the Bio.Entrez module and the MMCIF structure parser
  \item XXX include updates since end of May!
  \end{itemize}
}

%%%%%%%%%%%%%%%%%%%%%%%%%%%%%%%%%%%%%%%%%%%%%%%%%%%%%%%%%%%%%%%%%%%%%%%%%%%%%%%%

\section{Docker Images}
\frame
{
  \frametitle{Docker Images}

  \begin{itemize}
  \item Previously existing containers:
    \begin{itemize}
    \item basic container with Python 2 and 3, Biopython and all its dependencies
    \item BioSQL container
  \end{itemize}
  \item New: Jupyter notebook containers
    \begin{itemize}
      \item basic
      \item including Biopython tutorial as notebooks
    \end{itemize}
  \end{itemize}
}

%%%%%%%%%%%%%%%%%%%%%%%%%%%%%%%%%%%%%%%%%%%%%%%%%%%%%%%%%%%%%%%%%%%%%%%%%%%%%%%%

\section{General Updates}
\frame
{
  \frametitle{Website}

Our website has been migrated from MediaWiki to GitHub Pages and is now under version control. The
continuous integration process on GitHub has been enhanced by including external services like Landscape,
Quantified Code and Codecov to perform quality review, test coverage analysis and generation of quality
metrics.
}

%%%%%%%%%%%%%%%%%%%%%%%%%%%%%%%%%%%%%%%%%%%%%%%%%%%%%%%%%%%%%%%%%%%%%%%%%%%%%%%%

\section{Conclusion}
\frame
{
  \frametitle{Conclusion}

  \begin{itemize}
  \item Another great year!
  \end{itemize}
}

\section*{Acknowledgements}
\frame
{
  \frametitle{Thanks to...}

  \begin{itemize}
  \item Peter Cock
  \item Biopython Community
  \item Lao Saal (PhD supervisor)
  \item Lund University Faculty of Medicine (PhD Student Travel Grant)
  \end{itemize}
}

\frame
{
  \frametitle{Find us here!}

  \begin{center}
  Website:\\
  \url{http://biopython.org}\\
  Main code repository:\\
  \url{http://github.com/biopython/biopython}

  Mailing lists:

  \begin{itemize}
  \item General list: \url{biopython@biopython.org}
  \item Developers list: \url{biopython-dev@biopython.org}
  \end{itemize}

  \includegraphics[height=.2\textheight]{../abstract/biopython.jpg}
  \end{center}
}

\end{document}